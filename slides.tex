\documentclass{beamer}

\usepackage[T1]{fontenc}
\usepackage[utf8]{inputenc}
\usepackage{amsmath}
\usepackage{amssymb}
\usepackage{amsfonts}

\newcommand{\R}{\mathbb{R}}
\newcommand{\C}{\mathbb{C}}
\newcommand{\Z}{\mathbb{Z}}

\DeclareMathOperator{\Ima}{Im}
\DeclareMathOperator{\Rea}{Re}
\DeclareMathOperator{\Rez}{Rez}
\DeclareMathOperator{\ctg}{ctg}
\DeclareMathOperator{\ch}{ch}
\DeclareMathOperator{\sh}{sh}


%conjugat
\newcommand*\conj[1]{\overline{#1}}

%d de la derivata & integrale
\newcommand{\dd}{\; \mathrm{d}}

% = conform teor UH?
\newcommand{\EqConformCaz}{\overset{\frac{0}{0}}{\underset{\mathrm{L'H}}{=\joinrel=}} }
%\newcommand{\disc}[2]{\ensuremath{\mathcal{U}(#1;#2)}}


%%   doc styles
\linespread{1.1}


%%%%%%%%%%%%%%%%%%%%%%%%%%%%%%%%%%%%%%%%%%%%%%

\title[Teorema Reziduurilor si aplicatii]{Teorema Reziduurilor si aplicatii}
\author{\emph{Autor:} Tapalaga Ecaterina Simona \\ \emph{Conducator stiintific:} Prof. Dr. Salagean Grigore}

\institute{\textsc{Universitatea Babes-Bolyai Cluj-Napoca} \par
            \textsc{Facultatea de Matematica si Informatica} }
\date{Iulie 2013}

\begin{document}
    \begin{frame}
        \titlepage
    \end{frame}

    \begin{frame}{Despre Teorema reziduurilor}
        \begin{itemize}
            \item Teorema reziduurilor e o unealta puternica a analizei complexe.
            \item Este utilizata in evaluarea integralelor functiilor analitice pe contururi.
            \item Reduce calculul integralei la mult mai simpla evaluare a rezidurilor.
            \item Adesea poate fi folosita la calculul integralelor reale.
        \end{itemize}
    \end{frame}

    \begin{frame}{Structura lucrarii}
        \begin{itemize}
            \item Lucrarea ilustreaza asemenea aplicatii. Unele integrale prezentate nu pot fi evaluate prin metode elementare.
            \item Lucrarea este impartita in trei capitole:

            \begin{itemize}
                \item Notiuni introductive
                \item Teorema reziduurilor
                \item Aplicatii ale teoremei reziduurilor
            \end{itemize}

        \end{itemize}
    \end{frame}

    \begin{frame}{Aplicatii ale teoriei reziduurilor la calculul unor integrale definite reale }
        $(*)$ Fie integrala $\displaystyle I=\int_{0}^{2\pi} R(\sin x, \cos x) \dd x$, unde
        $R(u,v)$ este o functie rationala reala ce nu are poli pe cercul $u^2+v^2=1$ .

        \[
            \text{Atunci : } \int_{0}^{2\pi} R(\sin x , \cos x) \dd x =
            2\pi \sum_{z\in \mathcal{U}(0;1)} \Rez(f;z)
        \]
        \[
            \text{unde } f(z) = \frac{1}{z} R\left(\frac{z-\frac{1}{z}}{2i},\; \frac{z+\frac{1}{z}}{2} \right)
        \]
    \end{frame}

    \begin{frame}
        $(**)$ Fie integrala
        \[
            I = \int_{-\infty}^{\infty} f(x) e^{i \alpha x}\dd x
        \]
        unde $f=P/Q$ , $Q(x)\neq 0$ , $x \in \R$ , $grad\ P = k$ , $grad\ Q =p$,
        iar $p \geq k+1$ .

        Daca $\alpha > 0$, atunci:
        \[
            I = \int_{-\infty}^{\infty} f(x) e^{i \alpha x}\dd x
                =2 \pi i \sum_{\Ima\ z >0} \Rez(g;z)
        \]
        , unde $g(z) = f(z) e^{i \alpha z}$.

    \end{frame}

    \begin{frame}{Aplicatii concrete}
        Sa se calculeze integrala
        \[
            I = \int_{-\infty}^{\infty} \frac{\cos x}{x^2+a^2} \dd x, \text{ unde } a>0 .
        \]
        \begin{align*}
            \text{Fie } I_1 &= \int_{-\infty}^{\infty} \frac{\cos x}{x^2+a^2} \dd x \\
            \text{si }  I_2 &= \int_{-\infty}^{\infty} \frac{\sin x}{x^2+a^2} \dd x (= 0 \text{ pe ca e impara}) \\
            \text{ si fie } I &= I_1 + i I_2 \\
            \implies I &= \int_{-\infty}^{\infty} \frac{e^{ix}}{x^2+a^2} \dd x
        \end{align*}
    \end{frame}

    \begin{frame}
        \begin{align*}
            &\left .
                \begin{aligned}
                    P(x) &= 1 \\
                    Q(x) &= a^2 +x^2
                \end{aligned}
            \right \}
            \begin{aligned}
                grad\; Q &\geq grad\; P+1 \\
                2 &\geq 1
            \end{aligned}
            \\
            &f(z) = \frac{e^{iz}}{a^2 +z^2}
        \end{align*}

        $a^2 +z^2 = 0 \implies z_{1,2} = \pm i a$ , dar doar $z_1 = i a$
        pol de gradul I $\in$ semiplanul superior
        \[
            \implies I = 2 \pi i\; \Rez(f;z_1) = 2 \pi i\; \Rez(f;i a)
        \]
    \end{frame}
    \begin{frame}
        \[
            \Rez(f;i a) = \lim_{z \to ia} (z - i a) \frac{e^{iz}}{z^2+a^2}
                = \frac{e^{-a}}{z + i a} = \frac{e^{-a}}{2ia}
        \]
        \[
            \implies I = 2 \pi i \frac{e^{-a}}{2ia} = \frac{e^{-a}\pi}{a}
        \]
        \[
            I_1 = \Rea \; I \quad I_2=\Ima \; I
                \implies I_1 = \frac{e^{-a}\pi}{a}; \quad I_2 = 0
        \]
    \end{frame}
\end{document}