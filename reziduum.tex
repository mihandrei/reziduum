\documentclass[10pt,a4paper]{article}
\usepackage[utf8]{inputenc}
\usepackage[T1]{fontenc}
\usepackage{amsmath}
\usepackage{amsthm}
\usepackage{amssymb}
\usepackage{amsfonts}

\newtheorem{theorem}{Teorema}
\newtheorem{definition}{Definitie}
\newtheorem{notation}{Notatie}
\newtheorem{observation}{Observatie}
\newtheorem{consequence}{Consecinta}
\newtheorem{example}{Ex}

\newcommand{\R}{\ensuremath{\mathbb{R}}}
\newcommand{\C}{\ensuremath{\mathbb{C}}}
\newcommand*\conj[1]{\overline{#1}}

\begin{document}
\title{Teorema Reziduurilor}
\author{Tapalaga Ecaterina Simona}
\date{Iunie 2013}
\maketitle

%\begin{abstract}
%Abstractul aceastei lucrari
%\end{abstract}

\section{Notiuni introductive}

\begin{notation}
	\begin{align*}
		&\C && \text{ planul complex} \\
		C^* &:= \C \setminus \{0\}   \\
		C_\infty &:= \C\cup \{\infty\} \\
		&\mathcal{P}(\C) && \text{ multimea partilor lui } \C \\
		&\mathcal{P}(\R) && \text{ multimea partilor lui } \R \\
		\mathcal{U}(z_0;r) &:= \{z \in \C \colon |z - z_0| < r \} && \text{ discul cu centru in } z_0 \text{ si raza } r \\
		\dot{\mathcal{U}}(z_0;r) &:= \mathcal{U}(z_0;r) \setminus \{z_0\} && \text{ discul punctat de raza } r \text{ si centrul in } z_0 \\
		\mathcal{U}(z_0;r_1,r_2) &:= \{z \in \C \colon r_1 < |z - z_0| < r_2 \} && \text{ coroana circulara de centru } z_0 \text { si raze } r_1 , r_2 \\
		\mathcal{U}(z_0;r_2) &= \mathcal{U}(z_0;r_1,r_2) && \text{ cand } r_1 = 0\\
		\overline{\mathcal{U}}(z_0;r) &:= \{z \in \C \colon |z - z_0| \leq r \} && \text{ discul inchis de raza } r \text{ si centru } z_0 \\
		\partial\mathcal{U}(z_0;r) &:= \{z \in \C \colon |z - z_0| = r \} && \text{ bordura de raza } r \text{ si zentru } z_0
	\end{align*}
\end{notation}

\begin{definition}
	Fie $D$ o submultime a lui $\C$.
	Spunem ca este deschisa daca $\forall z \in D, \exists \mathcal{U}(z;r) \subset D$
\end{definition}

\begin{observation}
	Multimile $\emptyset$ si $\C$ se considera deschise
\end{observation}

\begin{definition}
	Fie $A$ o submultime a lui $\C$.
	Spunem ca $A$ este inchisa daca complementara ei $C(A)$ este deschisa
\end{definition}

\begin{definition}
	Fie $B$ o submultime a lui $\C$. Spunem ca $B$ este conexa daca si numai
	daca $\nexists D_1, D_2 \in \C$ multimi deschise a.i.
	\begin{align*}
		B \cap D_1 &\neq \emptyset \\
		B \cap D_2 &\neq \emptyset \\
		B \cap D_1 \cap D_2 &= \emptyset \\
		B \subset D_1 &\cup D_2
	\end{align*}
\end{definition}

\begin{definition}
	O submultime $B \subset \C$ este conexa daca $\nexists$ doua submultimi
	nevide disjuncte si deschise a lui $B$ a.i. reuniunea lor este egala cu $B$
\end{definition}

\begin{observation}
	Daca $B$ este o submultime conexa a lui $\C$ iar $B \cap A \cap	\overline{B}$,
	atunci $A$ este conexa. In particular, aderenta $\forall$ multimi conexe este conexa.
\end{observation}

\begin{definition}
	Fie $D$ o submultime a lui $\C$. $D$ se numeste domeniu daca este deschisa si conexa.
\end{definition}

\begin{notation}
	\begin{align*}
		\overline{A} && \text{ multimea punctelor aderente ale lui } A \text{ inchiderea} \\
		A' && \text { multimea punctelor de acumulare ale lui } A \\
		fr(A) = \partial A && \text { frontiera lui } A = \overline{A} \cap \overline{CA}
	\end{align*}
\end{notation}

\begin{definition}
 	O  submultime $A$ a lui $\C$ se numeste marginita daca $\exists$ un disc
 	$\mathcal{U}(z_0;r)$ a.i. $A \subset \mathcal{U}(z_0;r)$
\end{definition}

\begin{definition}
	Un domeniu $D$ din $\C$ care pentru $\forall z \in D$ verifica
	$[z,z_0]\subset D$ se numeste domeniu stelat in raport cu $z_0 \in D$
\end{definition}

\begin{observation}\leavevmode
	\begin{enumerate}
		\item Un domeniu stelat in raport cu $\forall$ punct al sau se numeste
		 domeniu convex
		\item $\forall$ disc este stelat si convex
		\item Discurile punctate sunt domenii , dar nu sunt si stelate: nu exista
		 nici un punct $z_1 \in \dot{\mathcal{U}}(z_0;r)$ in raport cu care sa fie stelat
	\end{enumerate}
\end{observation}

\begin{definition}
	Spunem ca $A$ este compacta $\iff \forall$ sir $(z_u)$ din $A$ contine un
	subsir $(Z_{n_m})$ din $A$ care converge catre un punct $z_0 \in A$
\end{definition}

\begin{definition}
	Fie $A, B$ multimi din $\C$. Distanta de la $A$ la $B$ este
	\begin{equation}
		d(A,B) = \inf\{d(a,b)\colon a \in A , b \in B \}
	\end{equation}
\end{definition}

\begin{observation}\leavevmode
\begin{enumerate}
	\item Fie $A, B \subset \C, A \cap B \neq \emptyset$.

	Atunci $d(A, B) = 0$

	\item Fie $A, B \subset \C, A \cap B = \emptyset$.

	Atunci $d(A, B) \geq 0$

	\item Fie $A$ o multime compacta si $B$ o multime inchisa a.i. $A \cap B = \emptyset$.

	Atunci $d(A, B) > 0$
\end{enumerate}
\end{observation}

\begin{consequence}
	Fie $G$ o multime deschisa si $K$ compacta , $K\subset G$.

	Atunci $d(K,\partial B)>0$
\end{consequence}

\begin{consequence}
	Fie $G$ o multime deschisa si $K$ compacta , $K\subset G$.
	Atunci $\exists \mathcal{U}(z_n;r), 0 < r < d(K, \partial G) ,
	z_n \in K , k = \overline{1,n}$ a.i discul compact
	$\overline{\mathcal{U}}(z_n;r) \subset G$ si $K \subset \bigcup_{k=1}^{n} \mathcal{U}(z_n;r)$
\end{consequence}

\begin{definition}
	Fie functia $f:A \mapsto C$, unde $A\subset\C$ multime deschisa . Functia f
	se numeste derivabila in $z_0$ daca $\exists$ si este finita urmatoarea
	limita:
	\begin{equation}
		\lim_{z\to z_0} \frac{f(z)-f(z_0)}{z-z_0}
	\end{equation}
	Daca $\exists$, aceasta se noteaza cu $f'(z_0)$ si se numeste derivata
	functiei $f$ in $z_0$
\end{definition}

\begin{definition}
	Spunem ca $f$ este olomorfa pe $A (A\subset\C\text{deschisa})$ daca este
	derivabila in orice punct din $A$.
	Notam cu $\mathcal{H}(A)$ multimea tuturor functiilor olomorfe pe $A$
\end{definition}

\begin{definition}
	Spunem ca functia $f$ este $\R$ diferentiabila (real-diferentiabila)
	in $z_0 = x_0 + iy_0 \in A$ daca functiile $u=Re f$ si $v = Im f $ sunt
	diferentiabile in $(z_0, y_0)$
\end{definition}

\begin{definition}
	Spunem ca functia $f$ este $\C$ diferentiabila (complex-diferentiabila)
	in $z_0 \in A$ daca $\exists$ un numar complex $N$ si o functie
	$g:A\setminus \{z_0\} \mapsto \C$ a.i. $lim_{z\to z_0} g(z) = 0 $ si
	$f(z) = f(z_0) + N(z-z_0) + g(z)(z-z_0) , z \in A\setminus \{z_0\}$
\end{definition}

\begin{observation}
	Functia $f$ este derivabila in $z_0 \iff f$ este $\C$ diferentiabila in $z_0$
\end{observation}

\begin{theorem}
	Cauchy - Riemann

	O functie $f:A \mapsto C$ este derivabila in punctul $z_0 \in A$ daca si numai daca

	\begin{enumerate}
		\item $f$ este $\R$ diferentiabila in $z_0$
		\item este satisfacut sistemul Cauchy-Riemann in $z_0$
		\begin{equation}
			\left .
				\begin{aligned}
					\frac{\partial u}{\partial x} (x_0,y_0) &=  \frac{\partial v}{\partial y}(x_0,y_0) \\
					\frac{\partial u}{\partial y} (x_0,y_0) &= - \frac{\partial v}{\partial x}(x_0,y_0)
				\end{aligned}
			\right \}
		\end{equation}
	\end{enumerate}
	unde $u=Re f , v = Im f $ si $z_0 = x_0 + i y_0$
\end{theorem}

\begin{example}
	Fie $f:\C \mapsto C, f(z) = az + b\conj z$. Sa ser determine $a,b \in \R$ a.i.
	$f$ sa fie derivabila in $z, \forall z \in \C$
	\begin{proof}[Rez]
		Cautam $u$ si $v$ 
		
		Fie $z=x+iy \implies f(z) = a(x+iy) + b(x-iy)$ 
		
		Deci $u=x(a+b)$ si  $v=y(a-b)$		

		\begin{displaymath}
			\left.
				\begin{aligned}
					\frac{\partial u}{\partial x} (x,y) &=  a+b \\
					\frac{\partial u}{\partial y} (x,y) &= 0    \\
					\frac{\partial v}{\partial x} (x,y) &= 0    \\
					\frac{\partial v}{\partial y} (x,y) &= a-b \\
				\end{aligned}
			\right \}
			\implies
			\left \{
				\begin{aligned}
					a+b &=0   \\
					-(a-b) &=0
				\end{aligned}
			\right.
			\iff
			\left \{
				\begin{aligned}
					a+b &=0   \\
					b-a &=0 \quad |_+ \\
					&\implies b = a = 0 
				\end{aligned}
			\right.						
		\end{displaymath}
		Deci $f(z) = 0$
	\end{proof}
\end{example}

\end{document}

