\documentclass[12pt,a4paper]{article}
\usepackage[utf8]{inputenc}
\usepackage[T1]{fontenc}
\usepackage{amsmath}
\usepackage{amsthm}
\usepackage{amssymb}
\usepackage{amsfonts}

\newtheorem{theorem}{Teorema}
\newtheorem{definition}{Definitie}
\newtheorem{notation}{Notatie}

\begin{document}
\title{Teorema Reziduurilor}
\author{Tapalaga Ecaterina Simona}
\date{Iunie 2013}
\maketitle

%\begin{abstract}
%Abstractul aceastei lucrari
%\end{abstract}

\section{Notiuni introductive}

\begin{notation}

\begin{equation}
	\begin{aligned}
		\mathbb C &= \text{planul complex}\\
		C^* &= \mathbb{C}\setminus \{0\} \\
		C_\infty &= \mathbb{C}\cup \{\infty\} \\
		\mathcal{P}(\mathbb{C}) &= \text{multimea partilor lui } \mathbb{C} \\
		\mathcal{P}(\mathbb{R}) &= \text{multimea partilor lui } \mathbb{R} \\
		\mathcal{U}(z_0;r) &= \{z \in \mathbb{C} \colon |z - z_0| < r \} \text{discul cu centru in } z_0 \text{ si raza } r \\
		\dot{\mathcal{U}}(z_0;r) &= \mathcal{U}(z_0;r) \setminus \{z_0\}
	\end{aligned}
\end{equation}

		
\end{notation}


\end{document}

