Aceasta lucrare prezinta aplicatii ale teoremei reziduurilor atat la 
calulul unor integrale si serii, cat si la determinarea numarului 
de zerourilor a unor functii.

Lucrarea este alcatuita din trei capitole. Primul capitol reprezinta 
scheletul teoretic necesar pentru a putea introduce notiunea de reziduuri. 
Acesta cuprinde notiunile de integrala complexa, zerourile unei functii 
olomorfe si indexul unei curbe. 

Urmatorul capitol enunta teorema 
reziduurilor impreuna cu demonstratia ei si calcularea reziduului intr-un pol.
Acesta teorema are aplicatii variate si importante.
Ultimul capitol, cel mai cuprinzator, prezinta unele dintre ele.
Teorema reziduurilor ofera metode elegante de calcula unele integrale reale.
Se poate utiliza deasemenea si in calcularea unor sume de serii, care in 
general sunt greu de determinat clasic.
O alta aplicatie interesanta este demonstratia teoremei fundamentale a algebrei.
Teorema lui Rouche este de asemenea un rezultat util al teoriei reziduurilor. 
Incheiem lucrarea cu aplicatii ale acesteia.