\section{Functii elementare}

\begin{definition}
    Constantele, functia identitate, functia exponentiala si cea logasritmica se numesc
    Functii elementare fundamentale. Functile elementare sunt obtinute din acestea prin
    aplicarea repetata de un numar finit de ori a operatiilor algebrice
    ($ + \times : \sqrt{\quad}$) si compunerea cu alte functii elementare.
\end{definition}

\begin{observation}\leavevmode
    Functia exponentiala complexa este $f:\C \mapsto \C^*$, unde $f(z):=e^z$ si
    $e^z:=e^x(\cos y + i\sin y)$ pentru $z=x+iy \in \C$

    Proprietati:
    
		\begin{enumerate}
				\item $f(z_1+z_2) = f(z_1) f(z_2), \forall z_1,z_2\in \C $
				\item $f(z+2\pi i) = f(z) , z\in \C, k\in \Z $
				\item $f(-z) = \frac{1}{f(z)}, z \in \C $
				\item $f$ este olomorfa pe $\C$ si $f^{n}(z) = f(z)
				,\forall z \in \C, n \in \mathbb{N} $
    \end{enumerate}    
\end{observation}

\begin{observation}\leavevmode
    \begin{enumerate}
        \item Functia exponentiala este periodica, avand perioada $T=2\pi i$; deci nu
        este injectiva dacat restransa la intervalul $\alpha, \alpha+2\pi$
        \[ I = \{z \in \C \colon \alpha < im\,z < \alpha + 2 \pi, \alpha \in \R \} \]
        \item Consideram $t \in \C^*$. Ecuatia $e^z=t$ are o infinitate de solutii in
        $\C$. deci, $e^z=t \iff \exists k\in\Z$ astfel incat
        \[z=\log | t | + i(arg\,t +2k\pi)\]
        unde $Arg t \in (\-pi, \pi]$ si este argumentul lui $t$

        Stim ca $a=Arg(t)$ este solutia unica din $(-\pi, \pi]$ a ecuatiei
        \[\cos a + i \sin a = \frac{t}{|t|}\]
        Din aceste solutii ajungem la aplicatia multivoca logaritm
        \[\log:\C^*\mapsto \mathcal{P}(\C)\]
        \[Log(z) = ln\,|z| + i\, arg\,z, z \in \C^*\]
        iar $Arg: \C^* \mapsto \mathcal{P}(\C)$ este aplicatia multivoca argument
        unde $Arg(z):= \{arg(z) + 2k \pi \colon k \in \Z\}$

        \item Inversa functiei exponentiale definita pe $\C\setminus(-\infty,0]$ este ramura
        principala  a aplicatiei $Log$ de mai sus
        \[log\,z = ln|z| + i arg(z), z\in\C\setminus(-\infty,0]\]
        \item Functia log este olomorfa pe $\C\setminus(-\infty,0]$
    \end{enumerate}
\end{observation}