\section{Derivata complexa}
\begin{definition}
	Fie functia $f:A \mapsto C$, unde $A\subset\C$ multime deschisa . Functia f
	se numeste derivabila in $z_0$ daca $\exists$ si este finita urmatoarea
	limita:
	\begin{equation}
		\lim_{z\to z_0} \frac{f(z)-f(z_0)}{z-z_0}
	\end{equation}
	Daca $\exists$, aceasta se noteaza cu $f'(z_0)$ si se numeste derivata
	functiei $f$ in $z_0$
\end{definition}

\begin{definition}
	Spunem ca $f$ este olomorfa pe $A (A\subset \C \text{deschisa})$ daca este
	derivabila in orice punct din $A$.
	Notam cu $\mathcal{H}(A)$ multimea tuturor functiilor olomorfe pe $A$
\end{definition}

\begin{definition}
	Spunem ca functia $f$ este $\R$ diferentiabila (real-diferentiabila)
	in $z_0 = x_0 + iy_0 \in A$ daca functiile $u=Re f$ si $v = Im f $ sunt
	diferentiabile in $(z_0, y_0)$
\end{definition}

\begin{definition}
	Spunem ca functia $f$ este $\C$ diferentiabila (complex-diferentiabila)
	in $z_0 \in A$ daca $\exists$ un numar complex $N$ si o functie
	$g:A\setminus \{z_0\} \mapsto \C$ a.i. $lim_{z\to z_0} g(z) = 0 $ si
	$f(z) = f(z_0) + N(z-z_0) + g(z)(z-z_0) , z \in A\setminus \{z_0\}$
\end{definition}

\begin{observation}
	Functia $f$ este derivabila in $z_0 \iff f$ este $\C$ diferentiabila in $z_0$
\end{observation}

\begin{theorem}
	Cauchy - Riemann

	O functie $f:A \mapsto C$ este derivabila in punctul $z_0 \in A$ daca si numai daca

	\begin{enumerate}
		\item $f$ este $\R$ diferentiabila in $z_0$
		\item este satisfacut sistemul Cauchy-Riemann in $z_0$
		\begin{equation}
			\left .
				\begin{aligned}
					\frac{\partial u}{\partial x} (x_0,y_0) &=  \frac{\partial v}{\partial y}(x_0,y_0) \\
					\frac{\partial u}{\partial y} (x_0,y_0) &= - \frac{\partial v}{\partial x}(x_0,y_0)
				\end{aligned}
			\right \}
		\end{equation}
	\end{enumerate}
	unde $u=Re f , v = Im f $ si $z_0 = x_0 + i y_0$
\end{theorem}

\begin{example}
	Fie $f:\C \mapsto C, f(z) = az + b\conj z$. Sa ser determine $a,b \in \R$ a.i.
	$f$ sa fie derivabila in $z, \forall z \in \C$
	\begin{proof}[Rez]
		Cautam $u$ si $v$ 
		
		Fie $z=x+iy \implies f(z) = a(x+iy) + b(x-iy)$ 
		
		Deci $u=x(a+b)$ si  $v=y(a-b)$		

		\begin{displaymath}
			\left.
				\begin{aligned}
					\frac{\partial u}{\partial x} (x,y) &=  a+b \\
					\frac{\partial u}{\partial y} (x,y) &= 0    \\
					\frac{\partial v}{\partial x} (x,y) &= 0    \\
					\frac{\partial v}{\partial y} (x,y) &= a-b \\
				\end{aligned}
			\right \}
			\implies
			\left \{
				\begin{aligned}
					a+b &= a-b \quad |_ -a + b   \\
					0 &=0
				\end{aligned}
			\right.
			\iff
			\left \{
				\begin{aligned}
					&b = 0   \\
					&a \in \R					
				\end{aligned}
			\right.						
		\end{displaymath}
		Deci $f(z) = az$
	\end{proof}
\end{example}
