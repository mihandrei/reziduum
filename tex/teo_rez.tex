\section{Teorema Reziduurilor}

\begin{theorem}
    Fie functia $f\in \mathcal{H}(G)$, unde $G \subset \C$ multime deschisa.
    Notam cu $\rho$  mutimea tuturor punctelor singulare izolate ale lui $f$
    Fie $\widetilde{G}:=G \cup S $, iar $\gamma$ un contur in $G$ omotop cu zero
     in $\widetilde{G}$
    \begin{align*}
        \text{Atunci suma: }
        &\sum_{z \in \widetilde{G}} {n(\gamma;z) Rez(f;z)}
        \text{ este finita si}  \\
        \int_{\gamma} f(z) \dd z = 2\pi i &\sum_{z \in \widetilde{G}} n(\gamma;z) Rez(f;z)
    \end{align*}
    \begin{proof}
        $\exists \varphi:[0;1]^2 \mapsto G$ deformare continuua,
        $k = \varphi ([0;1]^2) \subset \widetilde{G}$ compact.

        Fie
        \begin{align*}
            r &:= \frac{1}{2} \dd \left(k, \C \setminus \widetilde{G}\right)
            \\
            D &:= \bigcup_{z\in k} \mathcal{U}(z;r)
        \end{align*}

        $k \subset D \subset \overline{D} \subset \widetilde{G}$

        $\gamma$ omotop cu $0$ in $D$

        $\overline{D} \cap \rho$ finita
                $\implies \exists \{b_1, \dotsc, b_k \} = \overline{D} \cap \rho$

        Fie $\Pi_{k}(z)$ partea principala a dezvoltarii lui $f$ in $b_k$

        Deci, functia $\displaystyle g := f - \sum_{k=1}^{n} \Pi_k$
        olomorfa mai putin in $b_k$ admite o prelungire olomorfa $g_1$ la $D$ .
        \begin{align*}
            \int_{\gamma}  g &= \int_{\gamma} g_1 = 0 \\
                           g &= g_1 |_{D=\{b_1, \dotsc, b_k\}}\\
            \implies \int_{\gamma} f &= \sum_{k=1}^{n} \int_{\gamma} \Pi_k
        \end{align*}

        Calculam
        \[
            \int_{\gamma} \Pi_k \text{ , unde }
            \Pi_k(z) = \sum_{m=1}^{\infty} \frac{a^{(k)} - m }{(z - b_k)^m}
        \]

        Seria este uniform convergenta pe $\forall$ parte compacta din
        $\C \setminus \{b_a \} \implies$ uniform convergenta pe
        $\{ \gamma \} \implies$ putem integra termen cu termen si
        \[
            \int_{\gamma} \frac{\dd}{z-b_k} m = 0 , \forall m>1
        \]
        Functia $\displaystyle \frac{1}{(z-b_n)^m}$ admite primitiva si
        $\displaystyle
            \int_{\gamma} \frac{\dd z} {z - b_k} = 2 \pi i \cdot n(\gamma;b_n) \cdot a_{-1}^{(k)}
        $ deci
        \[
            \int_{\gamma} f = 2 \pi i \sum_{k=1}{n} n(\gamma; b_k) Rez(f;b_n)
        \]

        Trebuie sa mai aratam ca $\forall z_0 \in \widetilde{G} \setminus(D \cap \rho)
        \colon n(\gamma; z_0)\cdot Rez(f;z_0) = 0$

        Intr-adevar, daca pentru
        $z_0\in \widetilde{G} \setminus (D\cap\rho)$ avem
        $Rez(f;z_0) \neq 0 \implies z_0 \in \rho $, deci $z_0\notin D$ si
        \[
            n(\gamma;z_0) = \frac{1}{2 \pi i} \int_{\gamma}
            \frac{\dd \xi}{\xi - z_0} = 0
        \]
        caci $h(\xi) = \frac{1}{\xi - z_0}$ olomorfa pe $D$ si $\gamma$ omotop cu zero
        \[
            \implies \int_{\gamma} f = 2 \pi i \sum_{z\in \widetilde{G}} n(\gamma;z) \cdot Rez(f;z)
        \]

    \end{proof}
\end{theorem}

\section{Puncte singulare izolate}

\begin{definition}
    Fie $G\subset \C$ multime deschisa si $f\in\mathcal{H}(G)$. Punctul $z_0 \in \C$
    se numeste punct singular izolat pentru functia $f$ daca $z_0 \notin G$, dar
    $\exists p>0$ a.i
    $\mathcal{\dot{U}}(z_0;p)\subset G \implies f \in \mathcal{H}(\mathcal{\dot{U}}(z_0;p))$
\end{definition}

\begin{observation}
    De exemplu functiile $\frac{\sin(z)}{z}$ , $\frac{1}{z}$, $e^{\frac{1}{z}}$
    au singularitati izolate in $z=0$
\end{observation}

\begin{observation}
    Daca $z_0$ este un punct singular izolat pentru $f\in\mathcal{H}(G)$, iar
    $p>0$ a.i $\mathcal{\dot{U}}(z_0;p)\subset G$ , atunci $f$ admite o dezvoltare in
    serie Laurent de forma
    \[
        f(z) = \sum_{k=-\infty}^{\infty} a_{k}(z-z_0)^{k},\quad z\in \mathcal{\dot{U}}(z_0;p)
    \]
    Coeficientul $a_{-1}$ al termenului $(z-z_0)^{-1}$ se numeste reziduul functiei $f$
    in $z_0$ si se noteaza cu $a_{-1} = Rez(f;z_0)$
\end{observation}

\begin{definition}
    Fie $G \subset \C$ multime deschisa, $f\in\mathcal{H}(G)$ , iar $z_0$ punct singular
    izolat al functiei $f$. Spunem ca:
    \begin{enumerate}
        \item  $z_0$ este punct eliminabil daca $f$ se extinde olomorf la $\Omega \cup \{z_0\}$
        \item  $z_0$ este pol daca $lim_{z \to z_0} f(z) = \infty$
        \item  $z_0$ este punct esential izolat daca $\nexists$ limita a lui $f$ in $z_0$
        \item Un punct $z$ este regular pentru $f$ daca $z$ este eliminabil pentru $f$
        sau $f$ este derivabila in $z$
    \end{enumerate}
\end{definition}

\section{Calcularea reziduului intr-un pol}
\begin{enumerate}
    \item Daca $z_0$ este un pol de ordin $k$ pentru $f$ atunci
    \[
    Rez(f;z_0) = \frac{1}{(k-1)!}\lim_{z \to z_0} \left[(z-z_0)^k f(z) \right]^{(k-1)}
    \]
    \item In cazul unui punct singular esential reziduul se calculeaza cu ajutoril dezvoltarii
    in serie Laurent
    \item Intr-un punct regular reziduul este 0
\end{enumerate}

\section{Aplicatii ale teoriei reziduurilor la calculul unor integrale definite reale}

% tre sa fie tipul 1
\begin{tip}[1]
    Fie integrala $\displaystyle I=\int_{0}^{2\pi} R(\sin x, \cos x) \dd x$, unde
    $R(u,v)$ este o functie rationala reala ce nu are poli pe cercul $u^2+v^2=1$

    \[
        \text{Atunci } \int_{0}^{2\pi} R(\sin x , \cos x) \dd x =
        2\pi i \sum_{z\in \mathcal{U}(0;1)} Rez(f;z)
    \]
    \[
        \text{unde } f(z) = \frac{1}{z} R\left(\frac{z-\frac{1}{z}}{2i}, \frac{z+\frac{1}{z}}{2} \right)
    \]

    \begin{proof}
        Utilizand formulele lui Euler:
        \[
            \cos x = \frac{e^{ix}+e^{-ix}}{2}, \sin x = \frac{e^{ix}-e^{-ix}}{2i} x \in \R
        \]
        si substitutia $e^{ix}=z$, avem ca

        \begin{align*}
            \int_{0}^{2\pi} R(\sin x , \cos x) \dd x &=
                \int_{\partial \mathcal{U}(0;1)}
                    R\left(\frac{z-\frac{1}{z}}{2i}, \frac{z+\frac{1}{z}}{2} \right)
                \frac{\dd z}{iz}
            \implies
                        \\
                        \int_{0}^{2\pi} R(\sin x , \cos x) \dd x
                            &= -i \int_{\partial \mathcal{U}(0;1)} f(z) \dd z
            \overset{T.Rez}{\implies}
                        \\
                        \int_{\partial \mathcal{U}(0;1)} f(z) \dd z &=
                2\pi i \sum_{|z|<1} Rez(f;z)
            \implies
                        \\
                        \int_{0}^{2\pi} R(\sin x , \cos x) \dd x &=
                2\pi \sum_{|z|<1} Rez(f;z)
        \end{align*}


    \end{proof}
\end{tip}


\begin{tip}
    Fie $R$ o functie rationala reala , $R=P / Q$ unde $P$ si $Q$ polinoame de
    grad $n$ , respectiv $m$, $Q(x)\neq 0 \quad \forall x\in \R$,
    $\displaystyle \lim_{z\to\infty} z f(z) =0, (n \leq m-2)$

        Atunci
    \[
        \int_{-\infty}^{\infty} R(x) \dd x = 2\pi i \sum_{Im z = 0} Rez(f;z)
    \]
    \begin{proof}
        $\exists M,r_1 >0 $ a.i
        \[
            \left|\frac{P(x)}{Q(x)} \right| \leq \frac{M}{|x|^2} , \quad |x| \geq r_1
        \]
        \[
            \int_{r_1}^{\infty} \frac{1}{x^2} \dd x \text{ converge } \implies
            \int_{r_1}^{\infty} \frac{P(x)}{Q(x)} \dd x \text{ converge }
        \]
        Analog
        \[
           \int_{ - \infty}^{- r_1} \frac{P(x)}{Q(x)} \dd x \text{ converge }
        \]

                \[
                    \text{Dar } \frac{P}{Q} \text{ continuua pe } [- r_1, r_1] \implies
            \exists \int_{- r_1}^{r_1} \frac{P(x)}{Q(x)} \dd x
                \]

        \[
            \int_{- \infty}^{0} \frac{P(x)}{Q(x)} \dd x \text{ si }
            \int_{0}^{\infty} \frac{P(x)}{Q(x)} \dd x \text{ converg } \implies
            \int_{- \infty}^{\infty} \frac{P(x)}{Q(x)} \dd x \text{ converge }
        \]

        Fie $r>0$ suficient de mare astfel incat toti polii lui $f$ din semiplanul
        superior sa fie continuti in $\Omega_r$, unde $\Omega_r = \{z\in\C \colon |z| < r,\quad Im z > 0\}$ .

                Fie $\gamma_r(t) = r e^{\pi i t}, t\in[0;1], \gamma=[-r;r]\cup \gamma_r$.

                Atunci $\gamma = \partial \Omega_r$, iar $(\gamma) = \Omega_r$
        $\overset{T. Rez}{\implies}$

        \begin{equation*}
            \int_{\gamma} f(z) \dd z = 2 \pi i \sum_{z \in \Omega_r} Rez(f;z)
                = 2 \pi i \sum_{Im z >0} Rez(f;z)  \qquad (*)
        \end{equation*}

        Pe de alta parte
        \begin{equation*}
            \int_{\gamma} f(z) \dd z = \int_{\gamma_r} f(z) \dd z + \int_{-r}^{r} f(x) \dd x \qquad (**)
        \end{equation*}

        Din $(*)$ si $(**)$ trecand la limita $\implies$

        \[
            2 \pi i \sum_{Im z >0} Rez(f;z) = \lim_{r\to \infty} \int_{\gamma_r} f(z) \dd z
                + \int_{- \infty}^{\infty} f(x) \dd x
        \]

        \[
            \text{Dar, } \lim_{z\to\infty} z f(z)=0 \implies \lim_{r\to\infty} \int_{\gamma_r} f(z) \dd z = 0
        \]
        \[
            \implies \int_{- \infty}^{\infty} f(x) \dd x = 2 \pi i \sum_{Im z >0} Rez(f;z)
        \]
    \end{proof}
\end{tip}


\begin{tip}
    Fie $R$ o functie rationala reala de forma
             $\displaystyle R = \frac{P}{Q}$,
             $Q(x) \neq 0$ ,
             $ x \in \R$ ,
             grad $Q >$ grad  $P+1$ si
             $\displaystyle \lim_{|z|\to\infty} R(z) = 0  $

             Atunci
             \[
                    \int_{- \infty}^{\infty} R(x) e^{ix} \dd x \text{ converge si}
                    \int_{- \infty}^{\infty} R(x) e^{ix} \dd x = 2 \pi i \sum_{Im z >0} Rez(f;z)
              \]
            unde $f(z) = R(z) e^{iz}$ .

    \begin{proof}
        Fie $r>0$ suficient de mare a.i. toti polii functiei $f$ din semiplanul superior
        sa fie continuti in $D$, unde $D=\{z\in\C \colon |z| < r ;\ Im\ z >0\}$

        Fie $C = \partial D \implies C = [-r;r] \cup \gamma_r$
        \[
            \overset{T.Rez}{\implies} \int_{C} f(z) \dd z = 2 \pi i \sum_{Im\ z >0} Rez(f;z)
        \]

        \begin{displaymath}
            \left.
                \begin{aligned}
                    \text{Dar } \int_{C} f(z) \dd z
                        = \int_{-r}{r} f(x) \dd x + \int_{\gamma_r} f(z) \dd z \\
                        r \to \infty
                \end{aligned}
            \right \}
            \implies
        \end{displaymath}

        \begin{align*}
            \implies 2\pi i \sum_{Im\ z >0 } Rez(f;z)
                &= \int_{-\infty}{\infty} f(x) dx + \lim_{r\to\infty} \int_{\gamma_r} f(z) \dd z
            \\
                &= \int_{-\infty}^{\infty} \frac{P(x)}{Q(x)} e^{ix} \dd x
                   + \lim_{r\to\infty} \int_{\gamma_r} \frac{P(z)}{Q(z)} e^{iz} \dd z
        \end{align*}

        \[
            g(z) = \frac{P(z)}{Q(z)}
        \]
        deci,
        \[
            \lim_{z\to\infty}g(z) = 0 \overset{L.Jordan}{\implies}
            \lim_{r\to\infty} \int_{\gamma_r} g(z) e^{iz} \dd z = 0
        \]
        Asadar,
        \[
            \int_{-\infty}^{\infty} \frac{P(x)}{Q(x)} e^{ix} \dd x
                =2 \pi i \sum_{Im\ z >0} Rez(f;z)
        \]
    \end{proof}
\end{tip}

\begin{tip}
    Fie integrala
    \[
        I = \int_{-\infty}^{\infty} f(x) e^{i \alpha x}\dd x
    \]
    unde $f=P/Q$ , $Q(x)\neq 0$ , $x \in \R$ , $grad\ P = k$ , $grad\ Q =p$,
    iar $p \geq k+1$ .

    Daca $\alpha > 0$, atunci:
    \[
        I = \int_{-\infty}^{\infty} f(x) e^{i \alpha x}\dd x
            =2 \pi i \sum_{Im\ z >0} Rez(g;z)
    \]
    , unde $g(z) = f(z) e^{i \alpha z}$.

    \begin{proof}
        Observam ca
        $\displaystyle
            \exists \int_{-\infty}^{\infty} f(x) e^{i \alpha x}\dd x
        $
        si este convergenta.
        Intr-adevar, pentru ca $\displaystyle p\geq k+1 \implies \lim_{z\to\infty}f(z) = 0$.
        Dar $\displaystyle f'(z) = \frac{h(z)}{Q^2(z)}$, unde $h$ este un polinom
        de grad cel mult $k+p-1$.

        Fie $x_0$ zeroul lui $h$ de modul maxim $\implies f'(x)$
        are semn constant pentru $x>|x_0| \implies f(x) $ monotona pentru $x>|x_0|$.

        Fie $x_1,\ x_2 \in \R \text{ cu } x_2 > x_1 > |x_0|$
        \[
            \begin{aligned}
            \text{Cum } \lim_{z\to\infty} f(z) = 0 \implies
                &\text{ fie } f>0 \text{ si } \lim_{x\to\infty} f(x) = 0^+ , x >|x_0|\\
                &\text{ fie } f<0 \text{ si } \lim_{x\to\infty} f(x) = 0^- , x >|x_0|
            \end{aligned}
        \]
        Aplicand a doua teorema de medie din calculul integral
        $\implies \exists \xi \in (x_1;x_2)$ a.i.
        \[
            \int_{x_1}^{x_2} f(x) \cos \alpha x\ \dd x
                = f(x_1) \int_{x_1}^{\xi} \cos \alpha t\ \dd t
                + f(x_2) \int_{\xi}^{x_2} \cos \alpha t\ \dd t
        \]
        \[
            \implies \left | \int_{x_1}^{x_2} f(x) \cos \alpha x\ \dd x  \right |
                \leq \frac{2}{\alpha} |f(x_1)| + \frac{2}{\alpha} |f(x_2)|
        \]
        \[
            \text{Stiind ca } \lim_{z\to\infty} f(z) = 0 \implies
                \forall \epsilon > 0 \quad \exists \delta(\epsilon) > 0 \text{ a.i. } |f(x)|<\frac{\epsilon \alpha }{4}
            x > \delta(\epsilon)
        \]
        Deci,
        \[
            \left | \int_{x_1}^{x_2} f(x) \cos \alpha x\ \dd x  \right |
                \leq \frac{2}{\alpha} \big[ |f(x_1)|+ |f(x_2)| \big] < \epsilon ,
        \]
        \[
            x_2 > x_1 > \max\{|x_0|,\ \delta(\epsilon)\}
                \implies \int_{0}^{\infty} f(x)\cos \alpha x\ \dd x
                \text{ converge }
        \]
        Analog $\exists$ si converge
        \[
            \int_{0}^{\infty} f(x)\sin \alpha x \dd x
        \]
        \[
            \implies \int_{0}^{\infty} f(x) e^{i \alpha x} \dd x
        \]
        este deasemenea convergenta.

        Fie $\Omega_r = \{z \in \C \colon |z|<r ; Im\; z>0\}$ ce contine toti polii functiei
        $g$ din semiplanul superior
        \[
            \overset{T.Rez}{\implies} \int_{\partial \Omega_r} g(z) \dd z
                = 2 \pi i \sum_{z \in \Omega_r} Rez(g;z)
                = 2 \pi i \sum_{Im\; z > 0} Rez(g;z)
        \]
        Dar
        \[
            \int_{\partial \Omega_r} g(z) \dd z
                = \int_{-r}^{r} f(x)e^{i \alpha x} \dd x
                + \int_{\gamma_r} g(z) \dd z
        \]
        \[
            \overset{L.Jordan}{\implies} \lim_{r \to \infty} \int_{\gamma_r} g(z) \dd z = 0
        \]
        \[
            \overset{r\to\infty}{\implies} \int_{-\infty}^{\infty} f(x)e^{i \alpha x} \dd x
                = 2 \pi i \sum_{Im\; z > 0} Rez(g;z)
        \]
    \end{proof}

\end{tip}

\begin{aplicatie}[1]
    Sa se calculeze integrala
    \[
    I = \int_{-\infty}^{\infty} \frac{\dd x}{a^4+x^4}
    \]
    \begin{proof}

    Este o integrala de tipul II
    \begin{align*}
        &\left .
            \begin{aligned}
                P(x) &= 1 \\
                Q(x) &= a^4 +x^4
            \end{aligned}
        \right \}
        grad\; Q > grad\; P+2 \\
        & f(z) = \frac{a}{a^4 +x^4}
    \end{align*}
    \[
        a^4 +x^4 = 0 \implies z^4 = -a^4 = a^4 (\cos \pi + i\sin \pi)
    \]
    \[
        \implies z_k = a \left( \cos \frac{\pi+2k\pi}{4} + i \sin \frac{\pi+2k\pi}{4} \right)
            , k=\overline{0,3}
    \]
    \begin{align*}
        z_0 &= a \left(\frac{\sqrt 2}{2} + i\frac{\sqrt 2}{2} \right)
            = \frac{a}{\sqrt 2} (1+i) \\
        z_1 &= \frac{a}{\sqrt 2} (-1+i) \\
        z_2 &= \frac{a}{\sqrt 2} (-1-i) \\
        z_3 &= \frac{a}{\sqrt 2} (1-i)
    \end{align*}
    \[
        I = 2 \pi i \sum_{Im \; z_k >0} Rez(f;z_k)
    \]
    \[
        \implies I = 2 \pi i [Rez(f;z_0) + Rez(f;z_1)]
    \]
    \[
        Rez(f;z_k) = \lim_{z \to z_k} (z-z_k) \frac{1}{z^4+a^4}
            \overset{\frac{0}{0}}{\underset{\mathrm{U H}}{=}}
            \lim_{z \to z_k} \frac{1}{4z^3}
            = - \frac{z_k}{4a^4}
    \]
    Deci,
    \[
        I = 2 \pi i \left[ \frac{a}{\sqrt 2} (1+i -1 +i) \right]
            = \frac{2 \pi i \cdot a \cdot 2i}{\sqrt 2}
            \implies I = - \frac{4 \pi a}{\sqrt 2}
    \]
    \end{proof}
\end{aplicatie}

\begin{aplicatie}
    Sa se calculeze integrala
    \[
        I = \int_{-\infty}^{\infty} \frac{\cos x}{x^2+a^2} \dd x, \text{ unde } a>0
    \]
    \begin{proof}
        Este o integrala de tip III:
        \begin{align*}
            \text{Fie } I_1 &= \int_{-\infty}^{\infty} \frac{\cos x}{x^2+a^2} \dd x \\
            \text{si }  I_2 &= \int_{-\infty}^{\infty} \frac{\sin x}{x^2+a^2} \dd x (= 0 \text{ pe ca e impara}) \\
            \text{ si fie } I &= I_1+I_2 \\
            \implies I &= \int_{-\infty}^{\infty} \frac{e^{ix}}{x^2+a^2} \dd x
        \end{align*}

    \end{proof}
\end{aplicatie}