\subsection{Teorema Reziduurilor}

\begin{theorem}[Teorema Reziduurilor]
    Fie functia $f\in \mathcal{H}(G)$, unde $G \subset \C$ multime deschisa.
    Notam cu $S$  mutimea tuturor punctelor singulare izolate ale lui $f$.
    Fie $\widetilde{G}:=G \cup S $, iar $\gamma$ un contur in $G$ omotop cu zero
     in $\widetilde{G}$.
    \begin{align*}
        \text{Atunci suma: }
        &\sum_{z \in \widetilde{G}} {n(\gamma;z) \Rez(f;z)}
        \text{ este finita si}  \\
        \int_{\gamma} f(z) \dd z = 2\pi i &\sum_{z \in \widetilde{G}} n(\gamma;z) \Rez(f;z) \text{ .}
    \end{align*}
    \begin{proof}
        $\exists \varphi:[0;1]^2 \mapsto G$ deformare continuua,
        $k = \varphi ([0;1]^2) \subset \widetilde{G}$ compact.

        Fie
        \begin{align*}
            r &:= \frac{1}{2} \dd \left(k,\; \C \setminus \widetilde{G}\right)
            \\
            D &:= \bigcup_{z\in k} \mathcal{U}(z;r)
        \end{align*}

        $k \subset D \subset \overline{D} \subset \widetilde{G}$

        $\gamma$ omotop cu $0$ in $D$

        $\overline{D} \cap S$ finita
                $\implies \exists \{b_1, \dotsc, b_k \} = \overline{D} \cap S$

        Fie $\Pi_{k}(z)$ partea principala a dezvoltarii lui $f$ in $b_k$

        Deci, functia $\displaystyle g := f - \sum_{k=1}^{n} \Pi_k$
        olomorfa mai putin in $b_k$, admite o prelungire olomorfa $g_1$ la $D$ :
        \begin{align*}
            \int_{\gamma}  g &= \int_{\gamma} g_1 = 0 \\
                           g &= g_1 |_{D=\{b_1, \dotsc, b_k\}}\\
            \implies \int_{\gamma} f &= \sum_{k=1}^{n} \int_{\gamma} \Pi_k
        \end{align*}

        Calculam :
        \[
            \int_{\gamma} \Pi_k \text{ , unde }
            \Pi_k(z) = \sum_{m=1}^{\infty} \frac{a_{- m}^{(k)} }{(z - b_k)^m}  \text{ .}
        \]

        Seria este uniform convergenta pe $\forall$ parte compacta din
        $\C \setminus \{b_k \} \implies$ uniform convergenta pe
        $\{ \gamma \} \implies$ putem integra termen cu termen si
        \[
            \int_{\gamma} \frac{\dd z}{(z-b_k)^m}  = 0 , \forall m>1 \text{ .}
        \]
        Functia $\displaystyle \frac{1}{(z-b_n)^m}$ admite primitiva si
        $\displaystyle
            \int_{\gamma} \frac{\dd z} {z - b_k} = 2 \pi i \cdot n(\gamma;b_n) \cdot a_{-1}^{(k)}
        $ deci
        \[
            \int_{\gamma} f = 2 \pi i \sum_{k=1}^{n} n(\gamma; b_k) \Rez(f;b_n)  \text{ .}
        \]

        Trebuie sa mai aratam ca $\forall z_0 \in \widetilde{G} \setminus(D \cap S)
        \colon n(\gamma; z_0)\cdot \Rez(f;z_0) = 0$ .

        Intr-adevar, daca pentru
        $z_0\in \widetilde{G} \setminus (D\cap S)$ avem
        $\Rez(f;z_0) \neq 0 \implies z_0 \in S $, deci $z_0\notin D$ si
        \[
            n(\gamma;z_0) = \frac{1}{2 \pi i} \int_{\gamma}
            \frac{\dd \xi}{\xi - z_0} = 0
        \]
        caci $h(\xi) = \frac{1}{\xi - z_0}$ olomorfa pe $D$ si $\gamma$ omotop cu zero
        \[
            \implies \int_{\gamma} f = 2 \pi i \sum_{z\in \widetilde{G}} n(\gamma;z) \cdot \Rez(f;z) .
        \]

    \end{proof}
\end{theorem}

\subsection{Puncte singulare izolate}

\begin{definition}
    Fie $G\subset \C$ multime deschisa si $f\in\mathcal{H}(G)$. Punctul $z_0 \in \C$
    se numeste punct singular izolat pentru functia $f$ daca $z_0 \notin G$, dar
    $\exists p>0$ a.i
    $\mathcal{\dot{U}}(z_0;p)\subset G \implies f \in \mathcal{H}(\mathcal{\dot{U}}(z_0;p))$ .
\end{definition}

\begin{observation}
    De exemplu functiile $\frac{\sin(z)}{z}$ , $\frac{1}{z}$, $e^{\frac{1}{z}}$
    au singularitati izolate in $z=0$ .
\end{observation}

\begin{observation}
    Daca $z_0$ este un punct singular izolat pentru $f\in\mathcal{H}(G)$, iar
    $p>0$ a.i $\mathcal{\dot{U}}(z_0;p)\subset G$ , atunci $f$ admite o dezvoltare in
    serie Laurent de forma :
    \[
        f(z) = \sum_{k=-\infty}^{\infty} a_{k}(z-z_0)^{k},\quad z\in \mathcal{\dot{U}}(z_0;p) .
    \]
    Coeficientul $a_{-1}$ al termenului $(z-z_0)^{-1}$ se numeste reziduul functiei $f$
    in $z_0$ si se noteaza cu $a_{-1} = \Rez(f;z_0)$ .
\end{observation}

\begin{definition}
    Fie $G \subset \C$ multime deschisa, $f\in\mathcal{H}(G)$ , iar $z_0$ punct singular
    izolat al functiei $f$. Spunem ca:
    \begin{enumerate}
        \item  $z_0$ este punct eliminabil daca $f$ se extinde olomorf la $\Omega \cup \{z_0\}$;
        \item  $z_0$ este pol daca $lim_{z \to z_0} f(z) = \infty$;
        \item  $z_0$ este punct esential izolat daca $\nexists$ limita a lui $f$ in $z_0$;
        \item Un punct $z$ este regular pentru $f$ daca $z$ este eliminabil pentru $f$
        sau $f$ este derivabila in $z$.
    \end{enumerate}
\end{definition}

\subsection{Calcularea reziduului intr-un pol}
\begin{enumerate}
    \item Daca $z_0$ este un pol de ordin $k$ pentru $f$ atunci
    \[
    \Rez(f;z_0) = \frac{1}{(k-1)!}\lim_{z \to z_0} \left[(z-z_0)^k f(z) \right]^{(k-1)}.
    \]
    \item In cazul unui punct singular esential reziduul se calculeaza cu ajutorul dezvoltarii
    in serie Laurent.
    \item Intr-un punct regular reziduul este 0.
\end{enumerate}
