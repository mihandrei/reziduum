
\begin{aplicatie}[1]
    Sa se calculeze integrala
    \[
    I = \int_{-\infty}^{\infty} \frac{\dd x}{a^4+x^4}
    \]
    \begin{proof}

    Este o integrala de tipul II
    \begin{align*}
        &\left .
            \begin{aligned}
                P(x) &= 1 \\
                Q(x) &= a^4 +x^4
            \end{aligned}
        \right \}
        grad\; Q > grad\; P+2 \\
        & f(z) = \frac{a}{a^4 +x^4}
    \end{align*}
    \[
        a^4 +x^4 = 0 \implies z^4 = -a^4 = a^4 (\cos \pi + i\sin \pi)
    \]
    \[
        \implies z_k = a \left( \cos \frac{\pi+2k\pi}{4} + i \sin \frac{\pi+2k\pi}{4} \right)
            , k=\overline{0,3}
    \]
    \begin{align*}
        z_0 &= a \left(\frac{\sqrt 2}{2} + i\frac{\sqrt 2}{2} \right)
            = \frac{a}{\sqrt 2} (1+i) \\
        z_1 &= \frac{a}{\sqrt 2} (-1+i) \\
        z_2 &= \frac{a}{\sqrt 2} (-1-i) \\
        z_3 &= \frac{a}{\sqrt 2} (1-i)
    \end{align*}
    \[
        I = 2 \pi i \sum_{\Ima  \; z_k >0} \Rez(f;z_k)
    \]
    \[
        \implies I = 2 \pi i [\Rez(f;z_0) + \Rez(f;z_1)]
    \]
    \[
        \Rez(f;z_k) = \lim_{z \to z_k} (z-z_k) \frac{1}{z^4+a^4}
            \overset{\frac{0}{0}}{\underset{\mathrm{U H}}{=}}
            \lim_{z \to z_k} \frac{1}{4z^3}
            = - \frac{z_k}{4a^4}
    \]
    Deci,
    \[
        I = 2 \pi i \left[ \frac{a}{\sqrt 2} (1+i -1 +i) \right]
            = \frac{2 \pi i \cdot a \cdot 2i}{\sqrt 2}
            \implies I = - \frac{4 \pi a}{\sqrt 2}
    \]
    \end{proof}
\end{aplicatie}

\begin{aplicatie}
    Sa se calculeze integrala
    \[
        I = \int_{-\infty}^{\infty} \frac{\cos x}{x^2+a^2} \dd x, \text{ unde } a>0
    \]
    \begin{proof}
        Este o integrala de tip III:
        \begin{align*}
            \text{Fie } I_1 &= \int_{-\infty}^{\infty} \frac{\cos x}{x^2+a^2} \dd x \\
            \text{si }  I_2 &= \int_{-\infty}^{\infty} \frac{\sin x}{x^2+a^2} \dd x (= 0 \text{ pe ca e impara}) \\
            \text{ si fie } I &= I_1+I_2 \\
            \implies I &= \int_{-\infty}^{\infty} \frac{e^{ix}}{x^2+a^2} \dd x
        \end{align*}

        \begin{align*}
            &\left .
                \begin{aligned}
                    P(x) &= 1 \\
                    Q(x) &= a^2 +x^2
                \end{aligned}
            \right \}
            \begin{aligned}
                grad\; Q &\geq grad\; P+1 \\
                2 &\geq 1
            \end{aligned}
            \\
            &f(z) = \frac{e^{iz}}{a^2 +x^2}
        \end{align*}

        $a^2 +x^2 = 0 \implies z_{1,2} = \pm i a$ , dar doar $z_1 = i a$
        pol de gradul I $\in$ semiplanul superior
        \[
            \implies I = 2 \pi i\; \Rez(f;z_1) = 2 \pi i\; \Rez(f;i a)
        \]
        \[
            \Rez(f;i a) = \lim_{z \to ia} (z - i a) \frac{e^{iz}}{z^2+a^2}
                = \frac{e^{-ia}}{z + i a} = \frac{e^{-ia}}{2ia}
        \]
        \[
            \implies I = 2 \pi i \frac{e^{-ia}}{2ia} = \frac{e^{-a}\pi}{a}
        \]
        \[
            I_1 = \Rea \; I \quad I_2=\Ima \; I
                \implies I_1 = \frac{e^{-a}\pi}{a}; \quad I_2 = 0
        \]
    \end{proof}
\end{aplicatie}

\begin{theorem}
    Fie $f \in \mathcal{M}(\C)$ si $z_1 , \dotsc , z_k$ poli ai functiei $f$ cu
    reziduurile $u_1, \dotsc , u_k$.
    Daca $f(z) \neq 0$, $z \in \Z$, $z_j \notin \Z$, $j=1, \dotsc, k$,
    iar $f(z) = O(z^{-2}), z\to \infty$, atunci
    \[
        \sum_{-\infty}^{\infty} f(\varphi) = -\pi \sum_{j=1}^{k} \Rez ( \ctg \pi z \cdot f(z);z_j )
    \]
    \begin{proof}
        pag 95-98 carte portocala
    \end{proof}
\end{theorem}

\begin{aplicatie}
    Sa se calculeze \[ \sum_{n=1}^{\infty} \frac{1}{n^4 + 1}\]
    \begin{proof}
        Se vede ca
        \[
            \sum_{-\infty}^{\infty} \frac{1}{n^4 + 1} = 1 + 2 \sum_{n=1}^{\infty} \frac{1}{n^4 + 1}
        \]
        Fie $\displaystyle f(z) = \frac{1}{z^4+1}$, atunci $f\in \mathcal{\C}$ , cu polii simplii $\pm 1, \pm i$
        \[
            \Rez(f;z_n) = \lim{z \to z_n} \frac{1}{z^4+1}
                \overset{\frac{0}{0}}{\underset{\mathrm{L I H}}{=}} \lim_{z \to z_n} \frac{1}{4z^3}
                =\frac{z_n}{-4}
        \]
        \begin{align*}
          \sum_{-\infty}^{\infty} \frac{1}{n^4 + 1}
            &= - \pi \left[-\frac{1}{4} \ctg \pi + \frac{1}{4} \ctg(-\pi)
              - \frac{i}{4} \ctg i\pi + \frac{i}{4}\ctg(-i\pi)\right] \\
            &= \frac{\pi}{4}[\ctg \pi + \ctg \pi + i\ctg i\pi + i\ctg i\pi] \\
            &= \frac{\pi}{2} \ctg \pi + \frac{\pi}{2} \mathrm{cth}\; \pi
        \end{align*}
        \[
            \text{Deci, } 1 + 2 \sum_{n=1}^{\infty} \frac{1}{n^4 + 1}
                = \frac{\pi}{2} \ctg \pi + \frac{\pi}{2} \mathrm{cth}\; \pi
        \]
        \[
            \implies \sum_{n=1}^{\infty} \frac{1}{n^4 + 1}
                = \frac{\pi}{4} [\ctg \pi + \frac{\pi}{2} \mathrm{cth}\; \pi ] - \frac{1}{2}
        \]

    \end{proof}
\end{aplicatie}

\section{Calcularea unei integrale pe un arc de curba simplu si rectificabil, dar nu inchis}

    In acest caz putem incerca sa formam o curba inchisa $\gamma_0 \cup \gamma_1$
    a.i. sa poata sa se aplice teorema reziduurilor , iar integrala pe noua curba
    $\gamma=\gamma_0 \cup \gamma_1$ sa se poata calcula cu
    reziduuri direct sau sa aiba o relatie simpla cu integrala cautata.

    Daca integrala este improprie , fiind limita unei alte integrale
    \[
        \int_{\gamma_0}= \lim_{\gamma \to \gamma_0} \int_{\gamma}
    \]
    atunci si arcul adaugat va varia si vom putea calcula integrala improprie
    cunoscand limita $\int_{\gamma_1}$ si daca  suma reziduurilor din domeniu $G$
    variabil are limita cunoscuta:
    \[
        \int_{\gamma_0} f \dd z = - \lim \int_{\gamma_1} f \dd z + 2 \pi i \lim \sum \Rez (f;z)
    \]

    \begin{aplicatie}
        Sa se calculeze
        \[
            I = \int_{0}^{\infty} \frac{\cos ax}{\ch \pi x} \dd x , a \in \R
        \]
        \begin{proof}
            \[
                f(z) = \frac{\cos az}{\ch \pi z}
            \]
            Polii acestei functii sunt simpli , $z = (zk+1)\frac{i}{2}$, $k \in \Z$
            Pentru a evita seria de reziduuri care este divergenta alegem conturul

            IMAGINE

            Pe latura $z=R + i y \quad ( 0 \leq y \leq 1)$

            \begin{align*}
              \left| i \int_{0}^{1} \frac{\cos a (R+iy)}{\ch \pi (R+iy)} \dd y \right|
                 &= \left| i \int_{0}^{1} \frac{e^{ia(R+iy)} + e^{-ia(R+iy)}}
                                           {e^{\pi(R+iy)} + e^{-\pi(R+iy)}} \dd y \right|
              \\ &< \frac{\int_{0}^{1} (e^{-ay} + e^{ay}) \dd y}{e^{\pi R} - e^{-\pi R}}
                \to 0
            \end{align*}
            \[
                \text{Ramane : } 2 \int_{0}^{R} \frac{\cos ax}{\ch \pi x} \dd x
                    + \int_{0}^{R} \left[
                                    - \frac{\cos a(i+x)}{\ch \pi (i+x)}
                                    - \frac{\cos a(i-x)}{\ch \pi (i-x)}
                                   \right] \dd x
                \longrightarrow 2 \pi i \Rez \left(f;\frac{i}{2}\right)
            \]
            Stiind ca
            \begin{align*}
                \ch \pi( x \pm i) &= -\ch \pi x
                \\
                \cos a(x \pm i) &= \cos ax \cdot \ch a \mp \sin ax \cdot \sh a
            \end{align*}
            obtinem ca
            \[
                2(1+\ch a) \int_{0}^{R} \frac{\cos ax}{\ch \pi x} \dd x
                    \longrightarrow 2 \pi i \Rez \left(f;\frac{i}{2}\right)
            \]
            \[
                \Rez \left(f;\frac{i}{2} \right)
                    = \lim_{z\to\frac{i}{2}} \left( z- \frac{i}{2}\right) \frac{\cos az}{\ch \pi z}
                    = \frac{\cos \frac{ai}{2}}{\pi \sh \frac{\pi}{2}}
                    = \frac{\ch \frac{a}{2}}{\pi i \sh \frac{\pi}{2}}
                    = \frac{\ch \frac{a}{2}}{\pi i}
            \]
            \[
                \implies I = 2 \pi i \frac{\ch \frac{a}{2}}{\pi} = 2 \ch \frac{a}{2}
            \]
        \end{proof}

    \end{aplicatie}

    \section{Aplicatii la dezvoltari in serie}

    \begin{theorem}
        Fie $f(z)$ o functie mereomorfa ai carei poli formeaza un sir infinit $z_k \to \infty$
        si $D_n$ un domeniu marginit de o curba rectificabila $\gamma_n$ si care nu trece
        prin nici un pol $z_n$
        \[
            \text{Atunci } \int_{\gamma_n} f(z) \dd z = 2\pi i \sum_{k=1}^{n} \Rez (f;z_n)
        \]
    \end{theorem}

    \begin{observation}\leavevmode
        \begin{enumerate}
            \item Daca $n \to \infty, \gamma_n$ variaza a.i. $D_n$ tinde catre un domeniu
                ce cuprinde toti polii $a_n$. Daca integrala din membreul I are o limita finita,
                atunci obtinem suma seriei de Reziduuri $ \sum_{n=1}^{\infty}\Rez(f;z_n)$ insumata
                dupa domeniul $D_n$
            \item Daca indicele $k$ ia valorile $1,2,\cdots $ si $|z_k|$ sunt strict crescatoare
                $|z_1|<|z_2|< \cdots$ , a.i. intre 2 curbe consecutive sa se afle un singur pol,
                vom obtine suma seriei convergente $\sum_{n=1}^{\infty}\Rez(f;z_k)$
            \item Daca  $|z_n|$ si  $|z_{-n}|$ sunt crescatori vom putea obtine suma seriei
                convergente  $\Rez(f;z_0) + \sum_{k=1}^{n}[\Rez(f;z_k)+ \Rez(f;z_{-k})]$ adica
                \[
                    \sum_{-\infty}^{\infty} \Rez(f;z_n)
                        = \frac{1}{2 \pi i} \lim_{k\to \infty} \int_{\gamma_k} f(z) \dd z
                \]
            \item Fie $f(z)$ o functie mereomorfa avand polii de gradul I , $z_k \to \infty$ si
                $g(z)$ o functie uniforma cu un numar finit de puncte singulare $a_h$, diferite de
                $z_n$ . Fie $\gamma_n$ cu $n>n_0$ ce contine punctele $a_h$ in interiorul sau.
                Atunci pentru functia $f(z)\cdot g(z)$ avem ca
                \[
                    \Rez(f\cdot g;z_n)=g(z_k) \Rez(f;z_n)
                \]
                Formula din Obs 3 se transforma astfel
                \[
                    \sum_{-\infty}^{\infty} \Rez(f;z_n) g(z_n)
                        = \frac{1}{2 \pi i} \lim_{k\to \infty} \int_{\gamma_k} f(z)g(z) \dd z
                        - \sum_{k\in\C} \Rez(f\cdot g; a_h)
                \]
                A doua suma este nula pentru $g(z)$ functie intreaga
        \end{enumerate}
    \end{observation}

    \begin{aplicatie}
        Sa se calculeza integrala
        \[
            I = \int_{0}^{2 \pi} \frac{\dd x}{a + \cos x}, a>1
        \]
        \begin{proof}

        \end{proof}
    \end{aplicatie}