\subsection{Aplicatii ale teoriei reziduurilor la calculul unor integrale definite reale}

% tre sa fie tipul 1
\begin{tip}[1]
    Fie integrala $\displaystyle I=\int_{0}^{2\pi} R(\sin x, \cos x) \dd x$, unde
    $R(u,v)$ este o functie rationala reala ce nu are poli pe cercul $u^2+v^2=1$ .

    \[
        \text{Atunci : } \int_{0}^{2\pi} R(\sin x , \cos x) \dd x =
        2\pi \sum_{z\in \mathcal{U}(0;1)} \Rez(f;z)
    \]
    \[
        \text{unde } f(z) = \frac{1}{z} R\left(\frac{z-\frac{1}{z}}{2i},\; \frac{z+\frac{1}{z}}{2} \right)
    \]

    \begin{proof}
        Utilizand formulele lui Euler:
        \[
            \cos x = \frac{e^{ix}+e^{-ix}}{2}\; ,\quad \sin x = \frac{e^{ix}-e^{-ix}}{2i}\; , \quad x \in \R
        \]
        si substitutia $e^{ix}=z$, avem ca :

        \begin{align*}
            \int_{0}^{2\pi} R(\sin x , \cos x) \dd x &=
                \int_{\partial \mathcal{U}(0;1)}
                    R\left(\frac{z-\frac{1}{z}}{2i}, \frac{z+\frac{1}{z}}{2} \right)
                \frac{\dd z}{iz}
            \implies
                        \\
                        \int_{0}^{2\pi} R(\sin x , \cos x) \dd x
                            &= -i \int_{\partial \mathcal{U}(0;1)} f(z) \dd z
            \overset{T.Rez}{\implies}
                        \\
                        \int_{\partial \mathcal{U}(0;1)} f(z) \dd z &=
                2\pi i \sum_{|z|<1} \Rez(f;z)
            \implies
                        \\
                        \int_{0}^{2\pi} R(\sin x , \cos x) \dd x &=
                2\pi \sum_{|z|<1} \Rez(f;z) .
        \end{align*}


    \end{proof}
\end{tip}


\begin{tip}
    Fie $R$ o functie rationala reala , $R=P / Q$ unde $P$ si $Q$ polinoame de
    grad $n$ , respectiv $m$, $Q(x)\neq 0 \quad \forall x\in \R$,
    $\displaystyle \lim_{z\to\infty} z f(z) =0, (n \leq m-2)$ .

        Atunci :
    \[
        \int_{-\infty}^{\infty} R(x) \dd x = 2\pi i \sum_{\Ima z > 0} \Rez(f;z) .
    \]
    \begin{proof}
        $\exists M,r_1 >0 $ a.i :
        \[
            \left|\frac{P(x)}{Q(x)} \right| \leq \frac{M}{|x|^2} , \quad |x| \geq r_1
        \]
        \[
            \int_{r_1}^{\infty} \frac{1}{x^2} \dd x \text{ converge } \implies
            \int_{r_1}^{\infty} \frac{P(x)}{Q(x)} \dd x \text{ converge .}
        \]
        Analog :
        \[
           \int_{ - \infty}^{- r_1} \frac{P(x)}{Q(x)} \dd x \text{ converge .}
        \]

        \[
            \text{Dar } \frac{P}{Q} \text{ continua pe } [- r_1, r_1] \implies
            \exists \int_{- r_1}^{r_1} \frac{P(x)}{Q(x)} \dd x .
        \]

        \[
            \int_{- \infty}^{0} \frac{P(x)}{Q(x)} \dd x \text{ si }
            \int_{0}^{\infty} \frac{P(x)}{Q(x)} \dd x \text{ converg } \implies
            \int_{- \infty}^{\infty} \frac{P(x)}{Q(x)} \dd x \text{ converge }
        \]

        Fie $r>0$ suficient de mare astfel incat toti polii lui $f$ din semiplanul
        superior sa fie continuti in $\Omega_r$, unde $\Omega_r = \{z\in\C \colon |z| < r,\quad \Ima z > 0\}$ .

        Fie $\gamma_r(t) = r e^{\pi i t}, t\in[0;1], \gamma=[-r;r]\cup \gamma_r$.

        Atunci $\gamma = \partial \Omega_r$, iar $(\gamma) = \Omega_r$
        $\overset{T. Rez}{\implies}$

        \begin{equation*}
            \int_{\gamma} f(z) \dd z = 2 \pi i \sum_{z \in \Omega_r} \Rez(f;z)
                = 2 \pi i \sum_{\Ima z >0} \Rez(f;z)  \qquad (*)
        \end{equation*}

        Pe de alta parte :
        \begin{equation*}
            \int_{\gamma} f(z) \dd z = \int_{\gamma_r} f(z) \dd z + \int_{-r}^{r} f(x) \dd x \qquad (**)
        \end{equation*}

        Din $(*)$ si $(**)$ trecand la limita $\implies$

        \[
            2 \pi i \sum_{\Ima z >0} \Rez(f;z) = \lim_{r\to \infty} \int_{\gamma_r} f(z) \dd z
                + \int_{- \infty}^{\infty} f(x) \dd x
        \]

        \[
            \text{Dar, } \lim_{z\to\infty} z f(z)=0 \implies \lim_{r\to\infty} \int_{\gamma_r} f(z) \dd z = 0
        \]
        \[
            \implies \int_{- \infty}^{\infty} f(x) \dd x = 2 \pi i \sum_{\Ima z >0} \Rez(f;z) .
        \]
    \end{proof}
\end{tip}


\begin{tip}
    Fie $R$ o functie rationala reala de forma
             $\displaystyle R = \frac{P}{Q}$,
             $Q(x) \neq 0$ ,
             $ x \in \R$ ,
             grad $Q \geq $ grad  $P+1$ si
             $\displaystyle \lim_{|z|\to\infty} R(z) = 0  $

             Atunci
             \[
                    \int_{- \infty}^{\infty} R(x) e^{ix} \dd x \text{ converge si}
                    \int_{- \infty}^{\infty} R(x) e^{ix} \dd x = 2 \pi i \sum_{\Ima z >0} \Rez(f;z)
              \]
            unde $f(z) = R(z) e^{iz}$ .

    \begin{proof}
        Fie $r>0$ suficient de mare a.i. toti polii functiei $f$ din semiplanul superior
        sa fie continuti in $D$, unde $D=\{z\in\C \colon |z| < r ;\ \Ima\ z >0\}$ .

        Fie $C = \partial D \implies C = [-r;r] \cup \gamma_r$
        \[
            \overset{T.\Rez}{\implies} \int_{C} f(z) \dd z = 2 \pi i \sum_{\Ima\ z >0} \Rez(f;z)
        \]

        \begin{displaymath}
            \left.
                \begin{aligned}
                    \text{Dar } \int_{C} f(z) \dd z
                        = \int_{-r}^{r} f(x) \dd x + \int_{\gamma_r} f(z) \dd z \\
                        r \to \infty
                \end{aligned}
            \right \}
            \implies
        \end{displaymath}

        \begin{align*}
            \implies 2\pi i \sum_{\Ima\ z >0 } \Rez(f;z)
                &= \int_{-\infty}^{\infty} f(x) dx + \lim_{r\to\infty} \int_{\gamma_r} f(z) \dd z
            \\
                &= \int_{-\infty}^{\infty} \frac{P(x)}{Q(x)} e^{ix} \dd x
                   + \lim_{r\to\infty} \int_{\gamma_r} \frac{P(z)}{Q(z)} e^{iz} \dd z
        \end{align*}

        \[
            g(z) = \frac{P(z)}{Q(z)}
        \]
        deci,
        \[
            \lim_{z\to\infty}g(z) = 0 \overset{L.Jordan}{\implies}
            \lim_{r\to\infty} \int_{\gamma_r} g(z) e^{iz} \dd z = 0
        \]
        Asadar,
        \[
            \int_{-\infty}^{\infty} \frac{P(x)}{Q(x)} e^{ix} \dd x
                =2 \pi i \sum_{\Ima\ z >0} \Rez(f;z)
        \]
    \end{proof}
\end{tip}

\begin{tip}
    Fie integrala
    \[
        I = \int_{-\infty}^{\infty} f(x) e^{i \alpha x}\dd x
    \]
    unde $f=P/Q$ , $Q(x)\neq 0$ , $x \in \R$ , $grad\ P = k$ , $grad\ Q =p$,
    iar $p \geq k+1$ .

    Daca $\alpha > 0$, atunci:
    \[
        I = \int_{-\infty}^{\infty} f(x) e^{i \alpha x}\dd x
            =2 \pi i \sum_{\Ima\ z >0} \Rez(g;z)
    \]
    , unde $g(z) = f(z) e^{i \alpha z}$.

    \begin{proof}
        Observam ca
        $\displaystyle
            \exists \int_{-\infty}^{\infty} f(x) e^{i \alpha x}\dd x
        $
        si este convergenta.
        Intr-adevar, pentru ca $\displaystyle p\geq k+1 \implies \lim_{z\to\infty}f(z) = 0$.
        Dar $\displaystyle f'(z) = \frac{h(z)}{Q^2(z)}$, unde $h$ este un polinom
        de grad cel mult $k+p-1$.

        Fie $x_0$ zeroul lui $h$ de modul maxim $\implies f'(x)$
        are semn constant pentru $x>|x_0| \implies f(x) $ monotona pentru $x>|x_0|$.

        Fie $x_1,\ x_2 \in \R \text{ cu } x_2 > x_1 > |x_0|$
        \[
            \begin{aligned}
            \text{Cum } \lim_{z\to\infty} f(z) = 0 \implies
                &\text{ fie } f>0 \text{ si } \lim_{x\to\infty} f(x) = 0^+ , x >|x_0|\\
                &\text{ fie } f<0 \text{ si } \lim_{x\to\infty} f(x) = 0^- , x >|x_0|
            \end{aligned}
        \]
        Aplicand a doua teorema de medie din calculul integral
        $\implies \exists \xi \in (x_1;x_2)$ a.i.
        \[
            \int_{x_1}^{x_2} f(x) \cos \alpha x\ \dd x
                = f(x_1) \int_{x_1}^{\xi} \cos \alpha t\ \dd t
                + f(x_2) \int_{\xi}^{x_2} \cos \alpha t\ \dd t
        \]
        \[
            \implies \left | \int_{x_1}^{x_2} f(x) \cos \alpha x\ \dd x  \right |
                \leq \frac{2}{\alpha} |f(x_1)| + \frac{2}{\alpha} |f(x_2)|
        \]
        \[
            \text{Stiind ca } \lim_{z\to\infty} f(z) = 0 \implies
                \forall \varepsilon > 0 \quad \exists \delta(\varepsilon) > 0 \text{ a.i. } |f(x)|<\frac{\varepsilon \alpha }{4}
            \;,\; x > \delta(\varepsilon)
        \]
        Deci,
        \[
            \left | \int_{x_1}^{x_2} f(x) \cos \alpha x\ \dd x  \right |
                \leq \frac{2}{\alpha} \big[ |f(x_1)|+ |f(x_2)| \big] < \varepsilon ,
        \]
        \[
            x_2 > x_1 > \max\{|x_0|,\ \delta(\varepsilon)\}
                \implies \int_{0}^{\infty} f(x)\cos \alpha x\ \dd x
                \text{ converge }.
        \]
        Analog $\exists$ si converge :
        \[
            \int_{0}^{\infty} f(x)\sin \alpha x \dd x
        \]
        \[
            \implies \int_{0}^{\infty} f(x) e^{i \alpha x} \dd x
        \]
        este deasemenea convergenta.

        Fie $\Omega_r = \{z \in \C \colon |z|<r ; \Ima\; z>0\}$ ce contine toti polii functiei
        $g$ din semiplanul superior.
        \[
            \overset{T.Rez}{\implies} \int_{\partial \Omega_r} g(z) \dd z
                = 2 \pi i \sum_{z \in \Omega_r} \Rez(g;z)
                = 2 \pi i \sum_{\Ima\; z > 0} Rez(g;z)
        \]
        Dar :
        \[
            \int_{\partial \Omega_r} g(z) \dd z
                = \int_{-r}^{r} f(x)e^{i \alpha x} \dd x
                + \int_{\gamma_r} g(z) \dd z
        \]
        \[
            \overset{L.Jordan}{\implies} \lim_{r \to \infty} \int_{\gamma_r} g(z) \dd z = 0
        \]
        \[
            \overset{r\to\infty}{\implies} \int_{-\infty}^{\infty} f(x)e^{i \alpha x} \dd x
                = 2 \pi i \sum_{\Ima\; z > 0} \Rez(g;z).
        \]
    \end{proof}

\end{tip}