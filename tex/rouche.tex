\subsection{Aplicatii in teoria functiilor}

Urmatorul rezultat face legatura intre numarul de zerouri si numarul de poli ai unei functii analitice.

\begin{theorem}
    Fie $D \subset \C$ domeniu stelat si $f \in \mathcal{M}(D)$ cu zerouurile :
    $a_1, \cdots , a_n \in D$, si polii $b_1, \cdots, b_m \in D$ .
    Atunci pentru orice contur $\gamma$ din $D$ ce evita toate zerourile
    si toti polii lui $f$ avem:
    \[
        \frac{1}{2\pi i} \int_{\gamma} \frac{f'}{f}(\zeta) \dd \zeta
        = \sum_{k=1}^{n} o(f; a_k)\; n(\gamma;a_k)
        + \sum_{l=1}^{m} o(f; b_l)\; n(\gamma;b_l)
    \]
\end{theorem}

O aplicatie a teoremei anterioare e teorema:

\begin{theorem}[Hurwitz]
    Fie $f_1, f_2, \cdots : D \mapsto \C$ un sir de functii ce converge local uniform
    la functia analitica $f : D \mapsto \C$. Daca $\forall i\;,\; f_i $ nu e identic nula pe D,
    atunci $f$ fie e identic nula, fie nu are nici un zerou in $D$.
\end{theorem}

\begin{theorem}
    Fie $D \in \C$ domeniu stelat si $f \in \mathcal{M}(D)$ cu zerouurile :
    $a_1, \cdots , a_n \in D$, si polii $b_1, \cdots, b_m \in D$.
    Notam:
    \begin{align*}
        N(0) &:= \sum_{k=1}^{n} o(f;a_k) && \text{numarul tuturor zerourilor lui } f \text{ ;} \\
        N(\infty) &:= - \sum_{l=1}^{m} o(f;b_l) && \text{numarul tuturor polilor lui } f \text{ ;}
    \end{align*}
    numarand multiplicitatile.  Fie $\gamma$ din $D$ ce inconjoara cu index 1 toate
    zerourile si toti polii. Atunci avem:
    \[
        \frac{1}{2\pi i} \int_{\gamma} \frac{f'}{f} (\zeta) \dd \zeta = N(0) - N(\infty) .
    \]
\end{theorem}

Daca $f$ nu are poli obtinem o formula pentru numarul de zerouri intr-un domeniu.

\begin{aplicatie}[Teorema fundamentala a algebrei]
    Orice polinom $P(z)$ de grad $n$ cu coeficienti complecsi, are exact n radacini complexe.
    \begin{proof}
        Deoarece $\lim_{|z| \to \infty} P(z) = \infty$ , $\exists R > 0$ a.i. $P$ nu are radacini
        $z$ cu $|z| \geq R$ . Numarul de zerouri a lui $P$ e :
        \[
            N(0) = \frac{1}{2\pi i}\int_{|\zeta| = R} \frac{P'(\zeta)}{P(\zeta)} \dd \zeta .
        \]
        Functia $P'/P$ are in $\infty$ un zerou simplu. Seria Laurent in $\infty$ e de forma :
        \[
            \frac{n}{z} + \frac{c_2}{z^2} + \frac{c_3}{z^3} + \cdots (n = grad(P)) .
        \]
        Deci:
        \[
            N(0) = n = grad(P).
        \]
    \end{proof}
\end{aplicatie}

\begin{theorem}[Rouche]
    Fie $f$, $g : D \mapsto \C$ analitice si $\gamma$ un contur din $D$ ce inconjoara
    orice punct din interiorul sau exact o data.
    Daca $|g(\zeta)| < |f(\zeta)| \forall \zeta \in \{\gamma\}$ atunci
    $f$, $f+g$ nu au zerouri in $\{\gamma\}$ si au in interiorul lui $\gamma$ acelasi numar de zerouri
    considerand multiplicitatile.
\end{theorem}

\begin{aplicatie}
    Sa se determine numarul solutiilor ecuatiei $z^4 -8z + 10 = 0 $ in $\mathcal{U}(0;1;3)$
    \begin{proof}
        \begin{center}
            \begin{tikzpicture}
                    \draw[help lines,->] (-3,0) -- (3,0) coordinate (xaxis);
                    \draw[help lines,->] (0,-3) -- (0,3) coordinate (yaxis);
                    \draw (0,0) circle (1cm);
                    \draw (0,0) circle (2cm);

                    \node[below] at (xaxis) {$Re$};
                    \node[left] at (yaxis) {$Im$};
                    \node[below left] at (1,0) {$1$};
                    \node[below left] at (2,0) {$2$};
                    \node[above left] at (0,1) {$i$};
                \end{tikzpicture}
        \end{center}

        \[
            \mathcal{U}(0;1;3) = \mathcal{U}(0;3)\setminus
                \big(\mathcal{U}(0;1) \cup \partial \mathcal{U}(0;1)\big)
        \]
        \begin{align*}
            N_1 :&= \text{ numarul solutiilor ecuatiei in } \mathcal{U}(0;3) \\
            N_2 :&= \text{ numarul solutiilor ecuatiei in } \mathcal{U}(0;1) \\
            N :&= \text{ numarul solutiilor ecuatiei in } \mathcal{U}(0;1;3)  \\
            N &= N_1 - N_2
        \end{align*}

        Determinam $N_1$. Avem $|z| < 3$.

        Alegem $f(z) = z^4$ si $g(z) = -8z + 10$.
        \begin{align*}
            |z^4| &= 3^4 = 81 \\
            |-8z + 10| &\leq 8|z| + 10 = 24 + 10 = 34 < 81 \implies \\
            |g(z)| &< |f(z)| \text{ pentru } |z| = 2 \overset{T.Rouche}{\implies} \\
            f(z) = 0 \text{ si } & f(z) + g(z) = 0 \text{ au acelasi numar de solutii in }
                \mathcal{U}(0;3) \\
            \implies & N_1 = 4
        \end{align*}

        Determinam $N_2$.
        \begin{align*}
            N'_2 :&= \text{ numarul solutiilor ecuatiei in } \mathcal{U}(0;1) \\
            N''_2 :&= \text{ numarul solutiilor ecuatiei pe } \partial \mathcal{U}(0;1) \\
            N_2 &= N'_2 + N''_2
        \end{align*}
        \begin{align*}
            |z| & = 1 \\
            f(z) & = -8z + 10 \implies g(z) = z^4 \\
            |f(z)| &\leq -8|z| + 10 =  18 \\
            |g(z)| &= |z^4| = 1 < 18 \overset{T.Rouche}{\implies}\\
            f(z) &= 0 \text{ si } f(z) + g(z) = 0 \text{ au acelasi numar de solutii }\\
            -8z + 10 &= 0 \implies z = 2 > 1 \implies N_2 = 0 \text{ numar de solutii in } \mathcal{U}(0;1)\\
            |f(z) + g(z)| &\geq |f(z)| + |g(z)| > 0, |z| = 1 \implies N''_2 = 0
        \end{align*}
        Deci $N = 4-0 = 4$
    \end{proof}
\end{aplicatie}

\begin{aplicatie}
    Fie $P_n(z) = a_0 + a_1 z + \cdots + a_n z^n$, $z \in \C$, unde $a_n \neq 0$ .
    \[
        \text{Fie } \alpha_n = \frac{ \sum_{k=0}^{n-1} |a_k|}{|a_n|}
             \text{ , si } \quad r > \max\{\alpha_n,1\} .
    \]
    Sa se arate ca toate solitiile polinomului $P_n \in \mathcal{U}(0;r)$.
    \begin{proof}
        Fie :
        \begin{align*}
            f(z) &:= a_n z^n \\
            g(z) &:= a_0 + a_1 z + \cdots + a_{n-1}z^{n-1} = P_n(z) - f(z)
        \end{align*}
        Avem :
        \begin{align*}
            |z| &= r \\
            |f(z)| &= a_n |r|^n = |a_n| r^n\\
            |g(z)| &= |a_0 + a_1 z + \cdots + a_{n-1}z^{n-1}| \\
                   &\leq |a_0| + |a_1| |z| + \cdots + |a_{n-1}| |z|^{n-1}\\
                   &= |a_0| + |a_1| r + \cdots + |a_{n-1}|r^{n-1}\\
            \text{Deoarece } & r^k \leq r^{n-1} \forall k = \overline{0,n-1} \\
            |g(z)| &\leq r^{n-1} (|a_0| + |a_1| + \cdots + |a_{n-1}|)
                = r^{n-1}\sum_{k=0}^{n-1}|a_k| \implies \\
            |g(z)| &\leq r^{n-1} \alpha_n |a_n|
                = \frac{\alpha_n r^n |a_n|}{r} = \frac{\alpha_n}{r} |f(z)| \\
            \text{Cum } & \frac{\alpha_n}{r} < 1 \text{ avem ca : }\\
            |g(z)| & < |f(z)| \;,\quad |z| = r
                \overset{T.Rouche}{\implies}\\
            f(z) = 0 & \text{ si } f(z) + g(z) = 0
                \text{ au acelasi numar de solutii in } \mathcal{U}(0;r)
        \end{align*}
        \[
            \left \{
                \begin{aligned}
                    f(z) &= 0 \\
                    P_n(z) &= 0
                \end{aligned}
            \right.
            \text{ au acelasi numar de solutii } \implies N = n
        \]

    \end{proof}
\end{aplicatie}