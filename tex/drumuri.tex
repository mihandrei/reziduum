\section{Drumuri in \C}

\begin{definition}
    Numim drum o functie $\gamma : [0,1] \mapsto \C$ continuua,
    unde $\gamma(0)=z_0$ si se numeste punct initial ,
    iar $\gamma(1)=z_1$ si se numeste punct final
\end{definition}

\begin{observation}
    Daca $z_0=z_1$ , atunci $\gamma$ este un drum inchis
\end{observation}

\begin{definition}
    Numim suportul drumului $\gamma$ , imaginea segmentului $[0,1]$ prin $\gamma$,
    adica $\gamma[0,1]$. Suportul se noteaza cu $\{\gamma\}$, unde
    $\{\gamma\} := \{\gamma(t) \in \C \colon t\in [0,1]\}$
\end{definition}

\begin{definition}
    Numim deformatie continuua a unui drum $\gamma_0:[0,1]\mapsto A$, $A\subset\C$
    in drumul $\gamma_1:[0,1]\mapsto A$ o functie $\varphi:[0,1]\times[0,1]\mapsto \C$
    cu proprietatea ca $\varphi(0, t) = \gamma_0(t)$ si $\varphi(1, t)=\gamma_1(t)$
\end{definition}

\begin{notation}
    $\mathcal{D}(z_0,z_1)$ drum care porneste din $z_0$ cu varful in $z_1$
\end{notation}

\begin{definition}
    Fie $\gamma_0$ si $\gamma_1 \in A$ doua drumuri. Spunem ca $\gamma_0$ este omotop
    cu $\gamma_1$ (in sens larg) si notam $\gamma_0 TODO \gamma_1$ daca $\exists$ o
    deformatie continuua in $A$ a lui $\gamma_0$ in $\gamma_1$
\end{definition}

\begin{definition}
    Fie $\gamma_0 \in \mathcal{D}(z_0, z_1)$ si $\gamma_1 \in \mathcal{D}(z_1, z_1)$ \\
    Definim drumul $(\gamma_0 \cup \gamma_1)$ astfel
    \[
        (\gamma_0 \cup \gamma_1)(t) =
        \left \{
		    \begin{aligned}
                \gamma_1(2t), t \in \left[0,\frac{1}{2}\right] \\
                \gamma_2(2t-1), t \in \left[\frac{1}{2},1\right]
	       	\end{aligned}
        \right .
    \]
    Acesta se numeste compunerea drumului $\gamma_0$ cu $\gamma_1$
\end{definition}


\begin{definition}
    Fie $\gamma$ drum din $\mathcal{D_{A}}(z_0,z_1)$ , iar $\gamma^-$ din
    $\mathcal{D}(z_1,z_0)$. Atunci $\gamma^-(t) = \gamma(1-t), t\in[0,1]$ si
    se numeste inversul drumului $\gamma$
\end{definition}

\begin{definition}
    Fie $z\in A$, iar $e_z(t)=z$ pt $t\in[0,1]$. Spunem ca $\gamma \in
    \mathcal{D_{A}}(z,z)$ este omotop cu zero daca $\gamma TODO e_z$
\end{definition}

\begin{definition}
    Un domeniu $D$ se numeste simplu conex daca $\forall$ drum inchis din acest
     domeniu este omotop cu zero in $D$
\end{definition}

\begin{observation}
    Daca $D$ este un domeniu din $\C$ stelat fata de unul dintre punctele sale,
    atunci spunem despre $D$ ca este simplu conex
\end{observation}